\section{Introduction}
This proposal aims to present ARIFI to the Global Navigation Satellite Systems Agency's (GSA) contest, named "Farming by Satellite 2020". The latter invites European and African young professionals to contribute to the farming sector, by putting into practice their experience in subjects related to Global Navigation Satellite Systems (GNSS) combined with their knowledge in agriculture.\\\\%
%
%
GNSSs provide users a wide variety of tools that can be applied in multiple ways. These go from PNT (positioning, navigation and timing) facilities to earth observation services and more.
%
Their use has proved to be increasingly helpful for a wide range of businesses. Many have rapidly integrated them to their operations and obtained great benefit from it.
% 
Just to point out some them, one can think of finances, that require a very precise track of time. Autonomous Cars as well as aircrafts, make use of satellite constellations together with augmentation systems, e.g Galileo and Egnos, to count with precise and reliable measurements of their speed and location. Many agricultors are advised by companies with acces to earth observation satellites to make weather forecasts and to appropriately select cultivation areas.
%
However, in some areas GNSS services potential has not been, which is tipically it has been  tipically mostly limited by user's lack of successful implementation ideas. 
%
The agriculture sector is not an exception to this, and it requires of professionals from different fields of study to invest time in developing new ideas seeking argiculture modernization and to provide farmers with tools that ease their day-to-day labours.

